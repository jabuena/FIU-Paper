\documentclass[10pt,sigconf]{acmart}

\usepackage{booktabs} % For formal tables

\graphicspath{{figure/}{figures/}}

% Copyright
%\setcopyright{none}
%\setcopyright{acmcopyright}
%\setcopyright{acmlicensed}
\setcopyright{rightsretained}
%\setcopyright{usgov}
%\setcopyright{usgovmixed}
%\setcopyright{cagov}
%\setcopyright{cagovmixed}


% DOI
\acmDOI{TBD}

% ISBN
\acmISBN{TBD}

%Conference
\acmConference[ICN'19]{ACM ICN}{Sept 2019}{Hong Kong, China} 
\acmYear{2019}
\copyrightyear{2019}

\acmPrice{15.00}


\begin{document}
\title{Who can you trust? \linebreak Onboarding techniques for NDNoT}
% \titlenote{Produces the permission block, and copyright information}
% \subtitle{Extended Abstract}

\author{Anonymous Anon}
% \authornote{Note}
% \orcid{1234-5678-9012}
% \affiliation{%
%   \institution{Affiliation}
%   \streetaddress{Address}
%   \city{City} 
%   \state{State} 
%   \postcode{Zipcode}
% }
% \email{email@domain.com}

% \author{Firstname Lastname}
% \orcid{1234-5678-9012}
% \affiliation{%
%   \institution{Affiliation}
%   \streetaddress{Address}
%   \city{City} 
%   \state{State} 
%   \postcode{Zipcode}
% }
% \email{email@domain.com}

% \author{Firstname Lastname}
% \orcid{1234-5678-9012}
% \affiliation{%
%   \institution{Affiliation}
% }
% \email{email@domain.com}

% \author{Firstname Lastname}
% \orcid{1234-5678-9012}
% \affiliation{%
%   \institution{Affiliation}
% }
% \email{email@domain.com}

% \author{Firstname Lastname}
% \orcid{1234-5678-9012}
% \affiliation{%
%   \institution{Affiliation}
% }
% \email{email@domain.com}


% The default list of authors is too long for headers}
% \renewcommand{\shortauthors}{F. Lastname et al.}


\begin{abstract}

  The rapid proliferation of sensors and their use in modern Internet of Things (IoT) environment has brought about a revolution in the way offices work.
  Researchers have identified the benefits that a data-centric approach like Named Data Networking can provide in such a scenario.
  The vast diversity of the sensors and the need for new services leads to the constant addition of newer devices making it an environment that is highly vulnerable to attacks. 
  This poster explores the different techniques for securely bootstrapping trust into devices and networks to potentially ~(a) ensure legitimate sensors / devices join the network and ~(b) the device identifies the correct network to join.
  An explorative effort in using a button-based approach to complete bootstrapping and the corresponding threat model and solution is discussed in this article.
  
\end{abstract}

% %
% % The code below should be generated by the tool at
% % http://dl.acm.org/ccs.cfm
% % Please copy and paste the code instead of the example below. 
% %
% \begin{CCSXML}
% <ccs2012>
%  <concept>
%   <concept_id>10010520.10010553.10010562</concept_id>
%   <concept_desc>Computer systems organization~Embedded systems</concept_desc>
%   <concept_significance>500</concept_significance>
%  </concept>
%  <concept>
%   <concept_id>10010520.10010575.10010755</concept_id>
%   <concept_desc>Computer systems organization~Redundancy</concept_desc>
%   <concept_significance>300</concept_significance>
%  </concept>
%  <concept>
%   <concept_id>10010520.10010553.10010554</concept_id>
%   <concept_desc>Computer systems organization~Robotics</concept_desc>
%   <concept_significance>100</concept_significance>
%  </concept>
%  <concept>
%   <concept_id>10003033.10003083.10003095</concept_id>
%   <concept_desc>Networks~Network reliability</concept_desc>
%   <concept_significance>100</concept_significance>
%  </concept>
% </ccs2012>  
% \end{CCSXML}

% \ccsdesc[500]{Computer systems organization~Embedded systems}
% \ccsdesc[300]{Computer systems organization~Redundancy}
% \ccsdesc{Computer systems organization~Robotics}
% \ccsdesc[100]{Networks~Network reliability}

% We no longer use \terms command
%\terms{Theory}

\keywords{Onboarding, Trust, Named Data Networking, Out-of-band}


\maketitle

\section{Introduction}

The widespread use of small, interconnected devices known as the Internet of Things (IoT) has generated a number of new possibilities in smart office environments in recent years. From temperature monitoring to motion tracking to the lights in the room, these IoT devices are becoming more prevalent in everyday situations. One of the core issues facing these devices, however, is the ability to securely integrate new devices into a pre-established network. This poster explores the different techniques for securely bootstrapping trust into devices and networks.



\section{Bootstrapping Taxonomy}

Out-of-band (OOB) channels are auxiliary to the main frequency of communication and are used as a means to bolster security.
OOB channels are characterized by the uncommon or unused frequencies that the protocols use to communicate or share any cryptographic information.
This section identifies techniques that can be used in the design of onboarding protocols highlighting the security properties and vulnerabilities for each of them. 

\subsection{Bluetooth Low Energy (BLE)}
\subsubsection{Device Provision Protocol (DPP)}
Seamless operations in a smart office involves communications among devices which are separated by small distances.
sThe functioning of DPP ~\cite{wifialliance} involves two devices: a configurator which broadcasts intent and an enrollee that responds to a broadcast when within the range.
After a response to the broadcast, the devices move to an auxiliary channel via BLE and exchange cryptographic information.
One of the main vulnerabilities of using BLE pairing is that it broadcasts messages openly to any device within range which makes them susceptible to Man in The Middle (MiTM) attacks.
BLE itself does not have a proof of possession of the bootstrapping keys in the auxiliary channel.
Any device within range can possibly eavesdrop on both device and obtain the bootstrapping keys.


%As a means for device onboarding, BLE is a viable alternative to Wifi in scenarios that do not require long range communication - not unlike an office. BLE lets a user send out a broadcast that states their intent to bootstrap with another device. The DPP requires two devices: a configurator that broadcasts intent and an enrollee within range that responds to a broadcast. When an enrollee does respond to a broadcast, the devices move to an auxiliary channel via BLE and exchange cryptographic information. One of the primary DPP characteristics is that the configurator sends out a request packet on the auxiliary channel and the enrollee must send back that same auxiliary channel request information to successfully onboard. One of the main vulnerabilities of using BLE pairing is its that broadcasts openly to any device within range. By broadcasting openly, the devices make themselves susceptible to Man in The Middle (MiTM) attacks. BLE itself does not have a proof of possession of the bootstrapping keys in the auxiliary channel. Any device within range can possibly eavesdrop on both device and obtain the bootstrapping keys.


\subsection{Buttons}
\subsubsection{Button Enabled Device Association (BEDA)}
The BEDA protocol ~\cite{soriente2007beda} is reliant on physical buttons on two devices.
These buttons are integral to the protocol because it uses the duration of a pressed button which is used to generate a shared secret.
However, this method presents a few vulnerabilities to replay attacks because hashes are not encrypted when they are sent out which lets an attacker pose as the sender by sending the hash to the original receiving device.



\subsection{Magnetometer Pairing}
Magnetometer pairing ~\cite{jin2014magpairing} is technique of bootstrapping that was designed for smartphones by making use of their magnetometers and a WiFi connection. Once the process has been initiated the devices will go through a standard Diffie-Hellman (DH) Key Exchange. During this process the devices will record their magnetometer readings at the same time. In order to foil eavesdroppers, the protocol makes use of the Interlock protocol to bolster its security against passive attacks. The protocol is secure against MiTM attacks because the Interlock the attacker will only receive half of an encrypted message of which is has no key to decrypt. The protocol is secure against Denial of Service (DoS) attacks because it is extremely difficult to overload the magnetometer readings without an abnormally large magnet. Magnetometer pairing makes use of the randomness of correlated sensor data by incorporating the lack of temporal or spatial alignment between devices into the encrypted secret. This randomness comes from the strength of the magnetic field at certain position, ambient noise, and device orientation.

\subsection{Audio Pairing}
\subsubsection{Human-Assisted Pure Audio Device Pairing (HAPADEP)}
The HAPADEP protocol ~\cite{soriente2008hapadep} makes use of the sonic frequency of sound waves by using two devices that have microphones and speakers. During the initial phase, the target device will send its public key to a controlling device by encoding the key using a fast codec that results in audio that is nonsensical to humans. Afterwards the protocol will enter a second verification phase where each device will encode the information that was retrieved in the last phase using a slow codec. This creates a human audible tune that can be used to confirm only intended devices were connected. One method to perform a DoS attack is to play loud audio to disrupt the recording during the first phase. HAPADEP can also be prone to impersonation attacks, but that would require an attacker to have knowledge of the codec being used to decode and encode the public keys. 

\subsubsection{Ultrasonic Pairing}
As an OOB channel, using an ultrasonic frequency ~\cite{mayrhofer2007security} can present some complications for the security of bootstrapping. In order to securely pair two devices, they must be located in the same room without any third-party devices or physical obstacles in the vicinity. If an attacker is in the same room, they would be able to mount a MiTM attack, impersonate another device, or eavesdrop on transmissions. The devices need to have clear line of sight with each other, otherwise it is possible that the sound waves will get distorted by physical obstacles or impersonated by an attacker.

\subsection{Vibration}
\subsubsection{SYNCVIBE}
SYNCVIBE ~\cite{lee2018syncvibe} is a method that uses vibration motors in smart devices for secure bootstrapping. The OOB channel is the physical vibration created by the motors and an accelerometer is used to decode the vibrations and receive the secret. In order for the devices to successfully pair with this method, they must have a primary channel to communicate in ala Wi-Fi or Bluetooth and must make physical contact with each other. This physical contact requirement is a way to notably reduce the possibility of MiTM attacks and eavesdroppers. The biggest impediments to vibration as a bootstrapping technique is the response time and lack of synchronization of smart device vibration motors. It is also possible for vibrations to be lost or altered during transit, corrupting the message.


\subsection{Light}
\subsubsection{LIRA/LIRA+}
The LIRA/LIRA+ ~\cite{kovavcevic2016flashing} protocols utilize visible light as the auxiliary OOB channel to exchange bootstrapping information while Bluetooth, Wifi, or some other in-band radio frequency is used to verify that the messages were received. In order to utilize a visible light channel, devices must contain a photodiode and light sensor. A controller serves as the light source where the devices are placed on, in order to transmit cryptographic information by flashing a sequence of lights that other devices can decrypt to get the shared secret. From there they can use the secret to create a secure communication channel. The main vulnerabilities with this method of bootstrapping is that third parties may be able to view the flashing sequence and inject or interfere with the sequences by flashing their own light from a distance. Doing so, however, would make attackers obvious to users but this shows the success of this method is dependent on the environment that it is in.

\section{Conclusion}

The aim of this poster was to explore the spectrum of out of band bootstrapping methods and what key characteristics they presented.
Each method’s security properties were surveyed, along with what sensors they would work best with.
By exploring these bootstrapping methods we have found many options that would work to ensure the security of IoT devices in a smart office environment. 


\bibliographystyle{acm}
\bibliography{refs} 

\end{document}
